
\section*{Question 1}
\begin{enumerate}[(a)]
  \item Let's define the variables that we need to use in the estimation.\\
  \begin{equation*}
  \begin{aligned}
    f(v_t,\theta) = & \begin{bmatrix}
      R_{t1} - \mu_1 \\
      R_{t2} - \mu_2 \\
      (R_{t1} - \mu_1)^2 - \sigma_1^2 \\
      (R_{t2} - \mu_2)^2 - \sigma_2^2
    \end{bmatrix} & , \quad 
    \theta = & \begin{bmatrix}
      \mu_1 \\
      \mu_2 \\
      \sigma_1^2 \\
      \sigma_2^2
    \end{bmatrix}  \\
    g_T(\theta) = & \quad \frac{1}{T} \sum_{t=1}^T f(v_t,\theta) & \\
  \end{aligned}
  \end{equation*}
We know from the lecture that we need to calculate the $\frac{\partial f}{\partial \theta'}$ to get the $\hat{D_T}$:
\begin{equation*}
  \begin{aligned}
    \frac{\partial f(v_t,\theta)}{\partial \theta'} = & \begin{bmatrix}
      -1 & 0 & 0 & 0 \\
      0 & -1 & 0 & 0 \\
      -2(R_{t1} - \mu_1) & 0 & -1 & 0 \\
      0 & -2(R_{t2} - \mu_2) & 0 & -1
    \end{bmatrix} \\\\
    \Rightarrow \hat{D_T} = & \frac{1}{T} \sum_{t=1}^T \frac{\partial f(v_t,\theta)}{\partial \theta'} = \begin{bmatrix}
      -1 & 0 & 0 & 0 \\
      0 & -1 & 0 & 0 \\
      0 & 0 & -1 & 0 \\
      0 & 0 & 0 & -1
    \end{bmatrix} = -I\\
  \end{aligned}
\end{equation*}
We also know that $A_T = I$ and $A_T g_T(\theta) = 0$. Therefore, we can calculate the $\hat{\theta}$:
\begin{equation*}
  \begin{aligned}
    A_T g_T(\theta) & = 0 \Rightarrow \quad g_T(\theta) =  0 
  \end{aligned}
\end{equation*}
\begin{equation*}
  \begin{aligned}
    g_T(\theta) = \begin{bmatrix}
      \frac{\sum R_{t1}}{T} - \mu_1 \\
      \frac{\sum R_{t2}}{T} - \mu_2\\
       \frac{\sum (R_{t1} - \mu_1)^2 }{T}- \sigma_1^2 \\
      \frac{\sum (R_{t2}- \mu_2)^2}{T}  - \sigma_2^2\\
  \end{bmatrix} = 0 \Rightarrow &  \begin{bmatrix}
      \frac{\sum_{t=1}^T R_{t1}}{T} \\
      \frac{\sum_{t=1}^T R_{t2}}{T} \\
      \frac{\sum_{t=1}^T (R_{t1} - \hat{\mu_1})^2}{T} \\
      \frac{\sum_{t=1}^T (R_{t2} - \hat{\mu_2})^2}{T}
    \end{bmatrix} = \begin{bmatrix}
      \hat{\mu_1} \\
      \hat{\mu_2} \\
      \hat{\sigma_1^2} \\
      \hat{\sigma_2^2}
    \end{bmatrix} = \hat{\theta} \\
  \end{aligned}
\end{equation*}

Our calculated $\hat{\theta}$ based on the given data is:
\begin{equation*}
  \hat{\theta} = \begin{bmatrix}
     0.0162 \\
     0.0045 \\
     0.0212 \\
     0.0167 \\
  \end{bmatrix}
\end{equation*}
\begin{center}
  \begin{lstlisting}[language=Python, caption=Python code for calculating $\hat{\theta}$, label={lst:q1a}, escapechar=|, frame=single, basicstyle=\small, showstringspaces=false, captionpos=b, breaklines=true, showspaces=false, showtabs=false, keywordstyle=\color{blue}, commentstyle=\color{gray}]
    mu_1 = sum(df['Stock1'])/len(df['Stock1'])
    mu_2 = sum(df['Stock2'])/len(df['Stock2'])
    sigma_1 = sum((df.Stock1 - mu_1)**2)/(len(df.Stock1))
    sigma_2 = sum((df.Stock2 - mu_2)**2)/(len(df.Stock2))
  \end{lstlisting}
\end{center}
\item Still we assume that there is no serial correlation in the moments. Therefore, we can calculate the $\hat{S_T}$ as follows:
\begin{equation*}
  \begin{aligned}
    \hat{S_T} = & \frac{1}{T} \sum_{t=1}^T f(v_t,\theta) f(v_t,\theta)'\\
    =& \frac{1}{T}\sum_{t=1}^T\scalemath{0.5}{ \begin{bmatrix}
      (R_{t1} - \mu_1)^2 & (R_{t1} - \mu_1)(R_{t2} - \mu_2) & (R_{t1} - \mu_1)^3 - \sigma_1^2(R_{t1} - \mu_1) & (R_{t1} - \mu_1)(R_{t2} - \mu_2)^2 - \sigma_2^2(R_{t1} - \mu_1)\\
      (R_{t1} - \mu_1)(R_{t2} - \mu_2) & (R_{t2} - \mu_2)^2 & (R_{t1} - \mu_1)^2(R_{t2} - \mu_2) - \sigma_1^2(R_{t2} - \mu_2) & (R_{t2} - \mu_2)^3 - \sigma_2^2(R_{t2} - \mu_2)\\
      (R_{t1} - \mu_1)^3 - \sigma_1^2(R_{t1} - \mu_1) & (R_{t1} - \mu_1)^2(R_{t2} - \mu_2) - \sigma_1^2(R_{t2} - \mu_2) & (R_{t1} - \mu_1)^4 - 2\sigma_1^2(R_{t1} - \mu_1)^2 + \sigma_1^4 & (R_{t1} - \mu_1)^2(R_{t2} - \mu_2)^2 - \sigma_1^2(R_{t1} - \mu_1)\sigma_2^2(R_{t2} - \mu_2)\\
      (R_{t1} - \mu_1)(R_{t2} - \mu_2)^2 - \sigma_2^2(R_{t1} - \mu_1) & (R_{t2} - \mu_2)^3 - \sigma_2^2(R_{t2} - \mu_2) & (R_{t1} - \mu_1)^2(R_{t2} - \mu_2)^2 - \sigma_1^2(R_{t1} - \mu_1)\sigma_2^2(R_{t2} - \mu_2) & (R_{t2} - \mu_2)^4 - 2\sigma_2^2(R_{t2} - \mu_2)^2 + \sigma_2^4\\
    \end{bmatrix}} \\
    & = {\begin{bmatrix}
      \hat{\sigma_1}^2 & Cov(\hat{R_1},\hat{R_2}) & \hat{m}_1^{(3)} & \hat{k}_{12}^{(2)}\\
      Cov(\hat{R_1},\hat{R_2}) & \hat{\sigma_2}^2 & \hat{k}_{21}^{(2)} & \hat{m}_2^{(3)}\\
      \hat{m}_1^{(3)} & \hat{k}_{12}^{(2)} & \hat{m}_1^{(4)} - \hat{\sigma}_1^4 & \hat{k}_{12}^{(3)}\\
      \hat{k}_{21}^{(2)} & \hat{m}_2^{(3)} & \hat{k}_{12}^{(3)} & \hat{m}_2^{(4)} - \hat{\sigma}_2^4\\
      \end{bmatrix}} \\
  \end{aligned}
\end{equation*}
Here the definition of $\hat{m}_1^{(j)}$ is the one in the lecture notes. For the $\hat{k}_{mn}^{(j)}$, I do not have a good way to write it in the matrix form. Therefore, I just write it as a way to sum up the terms in the matrix.\\

As we know that two process are independent, and normally distributed, we can calculate the $\hat{S_T}$ as follows:
\begin{equation*}
  \begin{aligned}
    \hat{S_T} = & \begin{bmatrix}
      \hat{\sigma_1}^2 & 0 & 0 & 0\\
      0 & \hat{\sigma_2}^2 & 0 & 0\\
      0 & 0 & 2\hat{\sigma_1}^4 & 0 \\
      0 & 0 & 0 & 2\hat{\sigma_2}^4 \\
    \end{bmatrix} \\
  \end{aligned}
\end{equation*}

Our calculated $\hat{S_T}$ based on the given data is:
\begin{equation*}
  \hat{S_T} = \begin{bmatrix}
    0.0212 & 0 & 0 & 0\\
    0 & 0.0167 & 0 & 0\\
    0 & 0 & 0.0347 & 0 \\
    0 & 0 & 0 & 0.0011 \\
  \end{bmatrix}
\end{equation*}
\begin{lstlisting}[language=Python, caption=Python function for calculating standard error of estimation , label={lst:q1a}, escapechar=|, frame=single, basicstyle=\small, showstringspaces=false, captionpos=b, breaklines=true, showspaces=false, showtabs=false, keywordstyle=\color{blue}, commentstyle=\color{gray}]
theta = np.array([mu_1,mu_2, sigma_1, sigma_2])
def f_v(theta,x):
    mu_1 = theta[0]
    mu_2 = theta[1]
    sigma_1 = theta[2]
    sigma_2 = theta[3]
    x_1 = x[0]
    x_2 = x[1]
    f = np.array([x_1-mu_1,x_2-mu_2,(x_1-mu_1)**2-sigma_1,(x_2-mu_2)**2-sigma_2]).reshape(len(theta),1)
    return f
def s(theta,x):
    f = f_v(theta,x)
    return f @ f.T
x = np.array([df[['Stock1','Stock2']]])[0]
s_hat = sum([s(theta,i) for i in x])/len(x)
# set non-diagonal elements to zero
s_hat = s_hat * np.eye(4)
\end{lstlisting}



\item Now we want to adjust the standard errors by Newey-West estimator. Therefore, we need to calculate the $\hat{\Gamma_1}$ by using the fact that two distributions are independent:
\begin{equation*}
  \begin{aligned}
    \hat{\Gamma_1} = & \scalemath{0.6}{ \begin{bmatrix}
      \frac{\sum^T_2 (R_{t}^1-\mu_1)(R_{t-1}^1-\mu_1)}{T-1} &0&  \frac{\sum^T_2 (R_{t}^1-\mu_1)(R_{t-1}^1-\mu_1)^2}{T-1} &0\\
    0& \frac{\sum^T_2 (R_{t}^2-\mu_2)(R_{t-1}^2-\mu_2)}{T-1} &0 & \frac{\sum^T_2 (R_{t}^2-\mu_2)(R_{t-1}^2-\mu_2)^2}{T-1}\\
    \frac{\sum^T_2 (R_{t-1}^1-\mu_1)(R_{t-1}^1-\mu_1)^2}{T-1} &0 &\frac{\sum^T_2 (R_{t-1}^1-\mu_1)^2(R_{t-1}^1-\mu_1)^2}{T-1}+\sigma_1^4 &0\\
    0 & \frac{\sum^T_2 (R_{t-1}^2-\mu_2)(R_{t-1}^2-\mu_2)^2}{T-1} & 0 & \frac{\sum^T_2 (R_{t-1}^2-\mu_2)^2(R_{t-1}^2-\mu_2)^2}{T-1}+\sigma_2^4\\ 
    \end{bmatrix}} 
  \end{aligned}
\end{equation*}
and then we can calculate the $\hat{S_T}$ as follows:
\begin{equation*}
  \begin{aligned}
    \hat{S_T} = & \begin{bmatrix}
      \hat{\sigma_1}^2 & 0 & 0 &0 \\
      0 & \hat{\sigma_2}^2 &0 &0 \\
      0 & 0 & 2\hat{\sigma_1}^4 & 0 \\
      0 & 0 & 0 & 2\hat{\sigma_2}^4 \\
    \end{bmatrix}
    + \frac{1}{2}(\hat{\Gamma}_1 + \hat{\Gamma}_1')\\
  \end{aligned}
\end{equation*}

Our calculated $\hat{S_T}$ based on the given data is:
\begin{equation*}
  \hat{S_T} = \begin{bmatrix}
    0.0162 & 0 & 0 & 0\\
    0 & 0.0160 & 0 & 0\\
    0 & 0 & 0.04 & 0 \\
    0 & 0 & 0 & 0.0013 \\
  \end{bmatrix}
\end{equation*}

\begin{lstlisting}[language=Python, caption=Python function for calculating Newy-West standard error , label={lst:q1a}, escapechar=|, frame=single, basicstyle=\small, showstringspaces=false, captionpos=b, breaklines=true, showspaces=false, showtabs=false, keywordstyle=\color{blue}, commentstyle=\color{gray}]
theta = np.array([mu_1,mu_2, sigma_1, sigma_2])
lag = 1
def gamma(theta,x,lag):
    gamma = {}
    for i in range(1,lag +1):
        lag = np.array(df[['Stock1','Stock2']].shift(i).dropna())
        tempt = []
        for num,j in enumerate(x[i:]):
            tempt.append(f_v(theta,j) @ f_v(theta,lag[num]).T)
        gamma[i] = sum(tempt)/len(tempt)
    gamma = [gamma[i] for i in gamma]
    return sum(gamma)

def s_newywest(theta,x):
    gamma_hat = gamma(theta,x,lag)
    return sum([s(theta,i) for i in x])/len(x) + 0.5 * (gamma_hat + gamma_hat.T)
s_hat_newywest = s_newywest(theta,x)
s_hat_newywest * np.eye(4)
\end{lstlisting}

\item  Now we want to compare the Sharpe ratio of two stocks. Therefore, we need to test the hypothesis that:
\begin{equation*}
  \begin{cases}
    H_0: \frac{\mu_1}{\sigma_1} = \frac{\mu_2}{\sigma_2} \\
    H_1: \frac{\mu_1}{\sigma_1} \neq \frac{\mu_2}{\sigma_2} \\
  \end{cases} \Rightarrow \begin{cases}
    H_0:  \mu_1\sigma_2 - \mu_2\sigma_1 = 0 \\
    H_1:  \mu_1\sigma_2 - \mu_2\sigma_1 \neq 0 \\
  \end{cases}
\end{equation*}
now we can define the $R(\theta)$ as follows:
\begin{equation*}
  \begin{aligned}
    R(\theta) = & \mu_1\sigma_2 - \mu_2\sigma_1 \\
  \end{aligned}
\end{equation*}
and then rewrite the hypothesis as follows:
\begin{equation*}
  \begin{aligned}
    H_0: & R(\theta) = 0 \\
    H_1: & R(\theta) \neq 0 \\
  \end{aligned}
\end{equation*}
Now we can use the Delta method to find the distribution of $R(\hat{\theta})$:
\begin{equation*}
  \begin{aligned}
    \sqrt{T}(R(\hat{\theta}) - R(\theta)) \xrightarrow{d} N(0,\frac{\partial R(\theta)}{\partial \theta'}V_{\theta}\frac{\partial R(\theta)}{\partial \theta'}) \\
    \sqrt{T}(R(\hat{\theta})) \xrightarrow{d} N(0,\frac{\partial R(\theta)}{\partial \theta'}V_{\theta}\frac{\partial R(\theta)}{\partial \theta'}) \\
  \end{aligned}
\end{equation*}
where $V_{\theta}$ is the variance of $\hat{\theta}$. Now we can calculate the $\frac{\partial R(\theta)}{\partial \theta'}$ as follows:
\begin{equation*}
  \begin{aligned}
    \frac{\partial R(\theta)}{\partial \theta'} = & \begin{bmatrix}
      \sigma_2 & -\sigma_1 & -\mu_2 & \mu_1
    \end{bmatrix} \\
  \end{aligned}
\end{equation*}
and we know that $V_{\theta} = \hat{S_T}$. Therefore, we can find the distribution of $R(\hat{\theta})$ as follows:
\begin{equation*}
\begin{aligned}
  \frac{\partial R(\theta)}{\partial \theta'}V_{\theta}\frac{\partial R(\theta)}{\partial \theta'} = & \begin{bmatrix}
    \sigma_2 & -\sigma_1 & -\mu_2 & \mu_1
  \end{bmatrix} \hat{S_T} \begin{bmatrix}
    \sigma_2 \\
    -\sigma_1 \\
    -\mu_2 \\
    \mu_1
  \end{bmatrix}  = \hat{V}_T
\end{aligned}
\end{equation*}

Now we can calculate the test statistic as follows:
\begin{equation*}
  \begin{aligned}
    T R(\hat{\theta})' \hat{V}_T^{-1} R(\hat{\theta}) \xrightarrow{d} \chi^2_1 \\
    \frac{T(\mu_1\sigma_2 - \mu_2\sigma_1)^2}{\hat{S}_T} \xrightarrow{d} \chi^2_1 \\
  \end{aligned}
\end{equation*}

Let's calculate the test statistic:
\begin{equation*}
  \begin{aligned}
    \hat{V}_T = \begin{bmatrix}
      0.0167 & -0.0212 & -0.0045 & 0.0162
    \end{bmatrix} \begin{bmatrix}
        0.0212 & 0 & 0 & 0\\
        0 & 0.0167 & 0 & 0\\
        0 & 0 & 0.0347 & 0 \\
        0 & 0 & 0 & 0.0011 \\
      \end{bmatrix} \begin{bmatrix}
        0.0167 \\
        -0.0212 \\
        -0.0045 \\
        0.0162
      \end{bmatrix} = 0.001449 \\
  \end{aligned}
\end{equation*}
therefore, the test statistic is $1.2602$ and the p-value is $0.2612$. Therefore, we cannot reject the null hypothesis under 5\% significance level. Therefore, we can conclude that there is no significant difference between the Sharpe ratios of two stocks.

\begin{lstlisting}[language=Python, caption=Python code for calculating the test statistics , label={lst:q1a}, escapechar=|, frame=single, basicstyle=\small, showstringspaces=false, captionpos=b, breaklines=true, showspaces=false, showtabs=false, keywordstyle=\color{blue}, commentstyle=\color{gray}]
  R_theta = mu_1*sigma_2 - mu_2*sigma_1
  R_prime = np.array([sigma_2, -sigma_1, -mu_2, mu_1])
  V_T = R_prime @ s_hat @ R_prime.T
  test_stat = len(df) * (R_theta)**2 / V_T
  \end{lstlisting}

\item Now we need to recalculate the standard error with the results in part (c). Our $\hat{V}_T$ is equal to $0.000005$ and test statistics is $3.1692$. The test statistics has been increased but it is still under the critical value of $3.841$, which means that we cannot reject the null hypothesis.

\item Now we use the bootstrap method to find a distribution of the Sharpe ratios. As we conducted the sample, we realized that our 95\% confidence interval contains zero, which means that still we cannot reject the null hypothesis and two stocks' Sharpe ratio are statistically indifferent.

\begin{lstlisting}[language=Python, caption=Python code for Bootstrapping , label={lst:q1a}, escapechar=|, frame=single, basicstyle=\small, showstringspaces=false, captionpos=b, breaklines=true, showspaces=false, showtabs=false, keywordstyle=\color{blue}, commentstyle=\color{gray}]
  # Function to calculate Sharpe ratio
def calculate_sharpe_ratio(returns):
    return np.mean(returns) / np.std(returns)

# Parameters
n_iterations = 10000
confidence_level = 0.95

# Calculate observed Sharpe ratio difference
sharpe_ratio_diff_observed = calculate_sharpe_ratio(R1) - calculate_sharpe_ratio(R2)

# Bootstrap procedure

sharpe_ratio_diff_bootstrap = []
np.random.seed(123)
for _ in range(n_iterations):
    # Generate bootstrapped samples
    bootstrap_sample_stock1 = np.random.choice(R1, size=len(R1), replace=True)
    bootstrap_sample_stock2 = np.random.choice(R2, size=len(R2), replace=True)

    # Calculate Sharpe ratios for bootstrapped samples
    sharpe_ratio_stock1 = calculate_sharpe_ratio(bootstrap_sample_stock1)
    sharpe_ratio_stock2 = calculate_sharpe_ratio(bootstrap_sample_stock2)

    # Store the Sharpe ratio difference
    sharpe_ratio_diff_bootstrap.append(sharpe_ratio_stock1 - sharpe_ratio_stock2)

# Calculate 95% confidence interval
lower_bound, upper_bound = np.percentile(sharpe_ratio_diff_bootstrap, [(1 - confidence_level) * 100 / 2, confidence_level * 100 - (1 - confidence_level) * 100 / 2])

print(f'Observed Sharpe Ratio Difference: {sharpe_ratio_diff_observed:.4f}')
print(f'95% Confidence Interval: [{lower_bound:.4f}, {upper_bound:.4f}]')
# The sharp ratio difference is not significant at 5% level.

\end{lstlisting}
\end{enumerate}
% \begin{enumerate}[(a)]
%     \item $f(v_t,\theta) = $
%     \[
% \begin{bmatrix}
%     R_{t1}-\mu_1 \\
%     R_{t2}-\mu_2\\
%     (R_{t1}-\mu_1)^2-\sigma_1^2 \\
%     (R_{t2}-\mu_2)^2-\sigma_2^2 \\
% \end{bmatrix}
% \]
% $\theta = $
%  \[
% \begin{bmatrix}
%     \mu_1 \\
%     \mu_2\\
%     \sigma_1^2 \\
%     \sigma_2^2 \\
% \end{bmatrix}
% \]
% $\frac{\partial f}{\partial \theta'}$
%  \[
% \begin{bmatrix}
%     -1&0&0&0 \\
%     0&-1&0&0\\
%     -2(R_{t1}-\mu_1)&0&-1&0 \\
%     0&-2(R_{t2}-\mu_2)&0&-1 \\
% \end{bmatrix}
% \]
% $\hat{D_T}=\frac{1}{T}\sum \frac{\partial f}{\partial \theta '}=$
%  \[
% \begin{bmatrix}
%     -1&0&0&0 \\
%     0&-1&0&0\\
%     0&0&-1&0 \\
%     0&0&0&-1 \\
% \end{bmatrix}
% \]
% $A_T=I$\\
% $A_T g_T(\theta)=0$\\
% $g_T(\theta) =0 =$
%  \[
% \begin{bmatrix}
%     \frac{\sum R_{t1}}{T} - \mu_1 \\
%     \frac{\sum R_{t2}}{T} - \mu_2\\
%      \frac{\sum (R_{t1} - \mu_1)^2 }{T}- \sigma_1^2 \\
%     \frac{\sum (R_{t2}- \mu_2)^2}{T}  - \sigma_2^2\\
% \end{bmatrix}
% \]
% therefore, \\
% $\hat{\mu_1} = \frac{\sum R_{t1}}{T}$
% $\hat{\mu_2} = \frac{\sum R_{t2}}{T}$
% $\hat{\sigma_1^2}=\frac{\sum (R_{t1}-\hat{\mu_1})^2}{T}$
% $\hat{\sigma_2^2}=\frac{\sum (R_{t2}-\hat{\mu_2})^2}{T}$

% \item 
% $f(v,\theta) f(v,\theta)'=$
% \[
% \begin{bmatrix}
%   R_{t1}-\mu_1 \\
%     R_{t2}-\mu_2\\
%     (R_{t1}-\mu_1)^2-\sigma_1^2 \\
%     (R_{t2}-\mu_2)^2-\sigma_2^2 \\
% \end{bmatrix}
% \begin{bmatrix}
%  R_{t1}-\mu_1 &
%     R_{t2}-\mu_2&
%     (R_{t1}-\mu_1)^2-\sigma_1^2 &
%     (R_{t2}-\mu_2)^2-\sigma_2^2 
% \end{bmatrix}
% \]
% given $R1 \sim N(\mu_1,\sigma_1^2)$ and $R2 \sim N(\mu_2,\sigma_2^2)$\\

% $\hat{S_T}=$
% \[
% \begin{bmatrix}
%   \hat{\sigma_1}^2 &0&0&0\\
%     0&\hat{\sigma_2}^2&0&0\\
%     0&0&2\hat{\sigma_1}^4&0 \\
%     0&0&0&2\hat{\sigma_2}^4 \\
% \end{bmatrix}
% \]

% \item 
% $f(v_t,\theta) f(v_{t-1},\theta)'=$
% \[
% \begin{bmatrix}
%   R_{t1}-\mu_1 \\
%     R_{t2}-\mu_2\\
%     (R_{t1}-\mu_1)^2-\sigma_1^2 \\
%     (R_{t2}-\mu_2)^2-\sigma_2^2 \\
% \end{bmatrix}
% \begin{bmatrix}
%  R_{t1-1}-\mu_1 &
%     R_{t2-1}-\mu_2&
%     (R_{t1-1}-\mu_1)^2-\sigma_1^2 &
%     (R_{t2-1}-\mu_2)^2-\sigma_2^2 
% \end{bmatrix}
% \]

% $\hat{\Gamma_1}=$
% % \resizebox{\textwidth}{!}{
% \[
% \begin{bmatrix}
%   \frac{\sum^T_2 (R_{t1}-\mu_1)(R_{t1-1}-\mu_1)}{T-1} &0&  \frac{\sum^T_2 (R_{t1}-\mu_1)(R_{t1-1}-\mu_1)^2}{T-1} &0\\
%     0& \frac{\sum^T_2 (R_{t2}-\mu_2)(R_{t2-1}-\mu_2)}{T-1} &0 & \frac{\sum^T_2 (R_{t2}-\mu_2)(R_{t2-1}-\mu_2)^2}{T-1}\\
%     \frac{\sum^T_2 (R_{t1-1}-\mu_1)(R_{t1-1}-\mu_1)^2}{T-1} &0 &\frac{\sum^T_2 (R_{t1-1}-\mu_1)^2(R_{t1-1}-\mu_1)^2}{T-1}+\sigma_1^4 &0\\
%     0 & \frac{\sum^T_2 (R_{t2-1}-\mu_2)(R_{t2-1}-\mu_2)^2}{T-1} & 0 & \frac{\sum^T_2 (R_{t2-1}-\mu_2)^2(R_{t2-1}-\mu_2)^2}{T-1}+\sigma_2^4\\  
% \end{bmatrix}
% \]
% % }

% $\hat{S_T} = $
% \[
% \begin{bmatrix}
%   \sigma_1^2 & 0 & 0 &0 \\
%   0 & \sigma_2^2 &0 &0 \\
%   0 & 0 & 2\sigma_1^4 & 0 \\
%   0 & 0 & 0 & 2\sigma_2^4 \\
% \end{bmatrix}
% +\frac{1}{2}(\hat{\Gamma_1}+\hat{\Gamma_1}')
% \]

% \item 
% $H_0: \mu_1*\sigma_2 - \mu_2*\sigma_1 = 0 $\\
% $R (\theta) = \mu_1*\sigma_2 - \mu_2*\sigma_1$\\
% $\theta = $
% \[
% \begin{bmatrix}
%   \mu_1 \\
%   \mu_2 \\
%   \sigma_1\\
%   \sigma_2 \\
% \end{bmatrix}
% \]
% $\frac{\partial R(\theta)}{\theta '}=$
% \[
% \begin{bmatrix}
%   \sigma_2 & -\sigma_1 & -\mu_2 & \mu_1
% \end{bmatrix}
% \]
% $T(R(\hat{\theta_T})'[\frac{\partial R(\theta)}{\theta '}\hat{V}\frac{\partial R(\theta)}{\theta '}']^{-1}(R(\hat{\theta}) \sim \chi^2$\\
% Sharpe Ratio Stock 1: 0.1113\\
% Sharpe Ratio Stock 2: 0.0352\\
% T-test Statistic: 1.4645\\
% p-value: 0.1433
% Fail to reject the null hypothesis: No significant difference in Sharpe ratios.
% \item 
% After  introducing the correlation matrix: \\
% Sharpe Ratio Stock 1: 0.1113\\
% Sharpe Ratio Stock 2: 0.0352\\
% Sample Correlation: 0.4187\\
% T-test Statistic: 12.5329\\
% Degrees of Freedom: 1198\\
% p-value: 0.0000\\
% Reject the null hypothesis: Stocks have different Sharpe ratios.
% \item 
% Observed Sharpe Ratio Difference: 0.0761\\
% 95\% Confidence Interval: [-0.0325, 0.1561]\\
% The sharp ratio difference is not significant at 5\% level.

% \end{enumerate}

