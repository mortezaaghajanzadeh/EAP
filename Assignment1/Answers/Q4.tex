
\section*{Question 4}
\begin{enumerate}[(a)]
    \item We predict the out-of-sample returns based on three different models:
    \begin{enumerate}[i.]
        \item Using dividend-price ratio:
            \begin{equation*}
                    \hat{R}_{t,DP}^e = \hat{\alpha} + \hat{\beta}_{t-1} dp_{t-1}  
            \end{equation*}
        \item Using all the variables from Dong et al. (2022):
            \begin{equation*}
                \hat{R}_{t,OLS}^e = \hat{\alpha} + \sum_{i=1}^{K}\hat{\beta}_{i,t-1} X_{i,t-1}
            \end{equation*}
        \item Using combination-mean forecast:
            \begin{equation*}
                \hat{R}_{t,CM}^e = \frac{1}{K}\sum_{i=1}^{K} \hat{R}_{t,i}^e
            \end{equation*}
            where, for each $i$,
            \begin{equation*}
                \hat{R}_{t,i}^e = \hat{\alpha}_i + \hat{\beta}_{i,t-1} dp_{t-1}
            \end{equation*}
    \end{enumerate}
    where $\hat{\alpha}$ and $\hat{\beta}$ are estimated using OLS regression for the in-sample data which is an expanding window from 1970/01 until the previous month of the out-of-sample period. The out-of-sample period is from 1985/01 until 2017/12. We compare each model using the out-of-sample $R^2$ which is defined as:
    \begin{equation*}
        R^2_{oc} = 1 - \frac{\sum_{t=1}^{T} (R_{t}^e - \hat{R}_{t}^e)^2}{\sum_{t=1}^{T} (R_{t}^e - \bar{R}_{t}^e)^2}
    \end{equation*}
    where $R_{t}^e$ is the realized excess return and $\hat{R}_{t}^e$ is the predicted excess return. The out-of-sample $R^2$ for each model :
    \begin{table}[H]
        \centering
        \begin{tabular}{c|c}
            Model & $R^2_{oc}$ \\
            \hline
            DP & -0.0237 \\
            OLS & -0.6856 \\
            CM & 0.0128 \\
        \end{tabular}
        \caption{Out-of-sample $R^2$ for each model}
        \label{tab:my_label}
    \end{table}
    As we can see, the out-of-sample $R^2$ for the DP and OLS model is negative which means that the DP model is not able to predict better than the benchmark which is the historical mean and the OLS model is worse than the DP model. However, the CM model has a positive out-of-sample $R^2$ which means that it is able to predict better than the benchmark. 

    \begin{lstlisting}[language=Python, caption=Python code for prediction, label={lst:q1a}, escapechar=|, frame=single, basicstyle=\small, showstringspaces=false, captionpos=b, breaklines=true, showspaces=false, showtabs=false, keywordstyle=\color{blue}, commentstyle=\color{gray}]
        data = pd.read_excel("Assignment1Data_G1.xlsx", sheet_name="Predictability")
        data = data.dropna()
        years = range(1970,2018)
        periods = [int(str(i) + "0" + str(j)) for i in years for j in range(1,10) if len(str(j)) == 1]
        periods.extend([int(str(i) + str(j)) for i in years for j in range(10,13) if len(str(j)) == 2 ])
        periods.sort()
        # %% Estimation
        def prediction(X,y):
            beta = sm.OLS(y,X).fit().params.to_numpy()
            return X.iloc[-1].to_numpy() @ beta
        BM_results = {}
        DP_results = {}
        OLS_results = {}
        CM_results = {}
        for prediction_period in tqdm([j for j in periods if j >= 198501]):
            in_sample_period = [i for i in periods if i < prediction_period]
            in_sample_data = data[data["Month"].isin(in_sample_period)]
            X = sm.add_constant(in_sample_data["dp"])
            y = in_sample_data["ExcessRet"]
            BM_results[prediction_period] = y.mean()
            DP_results[prediction_period] = prediction(X,y)
            columns = list(data)
            columns.remove('Month')
            columns.remove('ExcessRet')
            columns.remove('Rfree')
            columns.remove('dp')
            X = sm.add_constant(in_sample_data[columns])
            OLS_results[prediction_period] = prediction(X,y)
            CM_list= []
            for i in columns:
                X = sm.add_constant(in_sample_data[i])
                CM_list.append(prediction(X,y))
            CM_results[prediction_period] = np.mean(CM_list)
    \end{lstlisting}
    
    \item We need to perform the Diebold-Mariano test to test for the statistical significance of the difference between the out-of-sample $R^2$ of the models. The null hypothesis is that the difference between the out-of-sample $R^2$ is zero. The test statistic is defined as:
    
    \begin{equation*}
        DM = \frac{\bar{d}}{\sqrt{\frac{1}{T}\sum_{t=1}^{T}(d_t -\bar{d}_t)^2}}
    \end{equation*}
    where $d_t = \hat{\varepsilon}_t^2 - \tilde{\varepsilon}_t^2$ and $\bar{d}_t = \frac{1}{T_{os}\sum d_t}$. Also, we use the Clark-West test which has a different definition for the $d_t$:
    \begin{equation*}
        d_t = \hat{\varepsilon}_t^2 - [\tilde{\varepsilon}_t^2 -(\tilde{R}_t - \hat{R}_t)^2]
    \end{equation*}
    The calculated test statistics for each model are:
    \begin{table}[H]
        \centering
        \begin{tabular}{c|c|c}
            Model & $DM$ & $CW$\\
            \hline
            DP &  $-1.418$ & $-0.0606$\\
            OLS &  $-5.3738$ & $1.321$ \\
            CM & $0.5262$ & $2.0521$\\
        \end{tabular}
        \caption{Out-of-sample $R^2$ for each model}
        \label{tab:my_label}
    \end{table}

    The critical values for the Diebold-Mariano test are $-1.96$ and $1.96$ for the two-tailed test. Since the test statistics for the DP and CM models are not statistically significant, we cannot reject the null hypothesis that the difference between the out-of-sample $R^2$ is zero. On the other hand, the test statistic for the OLS model is statistically significant which means that the out-of-sample $R^2$ of the OLS model is statistically predict less than benchmark model due to the negative sign of the test statistic. 

    In addition, we can see that the test statistics for the Clark-West test are different from the Diebold-Mariano test. The critical values for the Clark-West test are the same as before. The test statistic for the DP and OLS model are not statistically significant which means that the null hypothesis cannot be rejected. However, the test statistic for the CM model is positive and statistically significant which means that  the model is statistically predicts better than benchmark model.

    \begin{lstlisting}[language=Python, caption=Python code for Diebold-Mariano test, label={lst:q1a}, escapechar=|, frame=single, basicstyle=\small, showstringspaces=false, captionpos=b, breaklines=true, showspaces=false, showtabs=false, keywordstyle=\color{blue}, commentstyle=\color{gray}]
        def DM_test(y_tilde, y_hat):
    T = len(y_hat)
    d = y_tilde**2 - y_hat**2
    delta_hat = np.mean(d)
    # sigma_hat = np.sqrt(np.sum((d - delta_hat)**2)/(T-1))
    # Newey-West correction with on lag
    sigma_hat = np.sqrt(np.sum((d - delta_hat)**2)/(T-1) + 2*np.sum([d[i]*d[i-1] for i in range(1,T)])/(T-1))

    DM = delta_hat/sigma_hat * np.sqrt(T)
    return DM

    def Clark_West_test(y_tilde, y_hat, R_tilde, R_hat):
    T = len(y_hat)
    d = y_tilde**2 - (y_hat**2 - (R_tilde - R_hat)**2)
    delta_hat = np.mean(d)
    # sigma_hat = np.sqrt(np.sum((d - delta_hat)**2)/(T-1))
    sigma_hat = np.sqrt(np.sum((d - delta_hat)**2)/(T-1) + 2*np.sum([d[i]*d[i-1] for i in range(1,T)])/(T-1))
    CW = delta_hat/sigma_hat * np.sqrt(T)
    return CW
    \end{lstlisting}

    \item Now we construct the portfolio based on each model based on the strategy in the question. 
\end{enumerate}