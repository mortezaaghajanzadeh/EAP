\section*{Question 3}
I just wrote a function that create the portfolios and calculate the portfolio return based on "Equal" and "Market" weighting. The function is shown in the code \ref{pcode:3a}. The function takes the following inputs:
\begin{itemize}
    \item \textbf{df}: The dataframe that contains the data
    \item \textbf{sorting\_car}: The variable that will be used to sort the stocks
    \item \textbf{number\_of\_portfolios}: The number of portfolios that will be created
    \item \textbf{weighting}: The type of weighting that will be used to calculate the portfolio return. The default is "Equal" weighting.
\end{itemize}

\begin{lstlisting}[language=Python, caption= Python function to create portfolios, label={pcode:3a}, escapechar=|, frame=single, basicstyle=\small, showstringspaces=false, captionpos=b, breaklines=true, showspaces=false, showtabs=false, keywordstyle=\color{blue}, commentstyle=\color{gray}]
    def get_portfolios(df, sorting_car, number_of_portfolios,weighting = 'Equal'):
    portfoli_df = df.dropna(subset=[sorting_car])[
        ['t', 'permno', sorting_car, 'me', 'ret']
    ].copy()
    portfoli_df['portfolios'] = portfoli_df.groupby('t')[sorting_car].transform(lambda x: pd.qcut(x, number_of_portfolios, labels=False)) 
    portfoli_df['portfolios'] = portfoli_df['portfolios'] + 1   # The highest value is the highest portfolio
    if weighting == 'market':
        portfoli_df['weight'] = portfoli_df.groupby(['t','portfolios'])['me'].transform(lambda x: x/sum(x))
        portfoli_df['ret'] = portfoli_df['ret'] * portfoli_df['weight']
    elif weighting == 'Equal':
        portfoli_df['weight'] = portfoli_df.groupby(['t','portfolios'])['me'].transform(lambda x: 1/len(x))
        portfoli_df['ret'] = portfoli_df['ret'] * portfoli_df['weight']
    return portfoli_df.groupby(['t','portfolios']).ret.sum().unstack().reset_index().rename(columns = {"t":"month"})
\end{lstlisting}

\begin{enumerate}[(a)]
\item Here is the result of the function for the "Equal" weighting. (Figure \ref{fig:3a})
\begin{figure}[htbp!]
    \centering
    \includegraphics[width=0.6\textwidth]{Out/3_1.pdf}
    \caption{Time series of the average returns of the portfolios based on the "Equal" weighting.}
    \label{fig:3a}
\end{figure}

\item Here is the result of the function for the "Market" weighting. (Figure \ref{fig:3b})
\begin{figure}
    \centering
    \includegraphics[width=0.6\textwidth]{Out/3_2.pdf}
    \caption{Time series of the average returns of the portfolios based on the "Market" weighting.}
    \label{fig:3b}
\end{figure}

\item Here we create the long-short portfolio. The long-short portfolio is created by taking the difference between the returns of the highest and the lowest portfolio. The result is shown in the figure \ref{fig:3c}.
\begin{figure}
    \centering
    \includegraphics[width=0.6\textwidth]{Out/3_3.pdf}
    \caption{Time series of the average returns of the long-short portfolio.}
    \label{fig:3c}
\end{figure}

Now we can test the CAPM, Fama-French 3 factors and the Fama-French 5 factors, Carhart, and HXZ models. You can find the function that I write to test the null hypothesis that $\alpha_{LS} =0$. I will get the p-value of the test.   


\begin{lstlisting}[language=Python, caption= Python function to run the test, label={pcode:3a}, escapechar=|, frame=single, basicstyle=\small, showstringspaces=false, captionpos=b, breaklines=true, showspaces=false, showtabs=false, keywordstyle=\color{blue}, commentstyle=\color{gray}]
    def time_series_regression(portfolios, factors, FactorModel):
    portfolios = portfolios.merge(factors, on='month', how='left')
    portfolios = portfolios.dropna()
    X = portfolios[FactorModel]
    X = sm.add_constant(X)
    Y = portfolios['long_short']
    model = sm.OLS(Y, X).fit(cov_type='HAC',cov_kwds={'maxlags':int(len(Y)**0.25)}) 
    pvalues = model.pvalues
    betas = model.params
    return [betas.iloc[0],pvalues.iloc[0]]
\end{lstlisting}

\begin{table}[htbp]
    \caption{$\alpha$ test for long-short portfolio with different models}
    \begin{tabularx}{\linewidth}{CC}
        \caption*{Equal Weighted }
        \begin{tabular}{lcc}
\toprule
 & $\alpha$ & $Pvalue$ \\
\midrule
CAPM & 0.010 & 0.000 \\
FF3 & 0.008 & 0.000 \\
CAR & 0.003 & 0.167 \\
FF5 & 0.003 & 0.167 \\
HXZ & 0.000 & 0.944 \\
\bottomrule
\end{tabular}

        &
        \caption*{Market Weighted }
        \begin{tabular}{lcc}
\toprule
 & $\alpha$ & $Pvalue$ \\
\midrule
CAPM & 0.004 & 0.073 \\
FF3 & 0.002 & 0.267 \\
CAR & -0.000 & 0.790 \\
FF5 & -0.003 & 0.061 \\
HXZ & -0.004 & 0.036 \\
\bottomrule
\end{tabular}

    \end{tabularx}
\end{table}
\item 

\begin{table}[htbp]
    \caption{$\alpha$ test long-short portfolio for in and out of sample with equal weighting}
    \begin{tabularx}{\linewidth}{CC}
        \caption*{Sample period }
        \begin{tabular}{lcc}
\toprule
 & $\alpha$ & $Pvalue$ \\
\midrule
CAPM & 0.010 & 0.000 \\
FF3 & 0.008 & 0.000 \\
CAR & 0.006 & 0.008 \\
FF5 & 0.005 & 0.016 \\
HXZ & 0.005 & 0.081 \\
\bottomrule
\end{tabular}

        &
        \caption*{Post-publication period}
        \begin{tabular}{lcc}
\toprule
 & $\alpha$ & $Pvalue$ \\
\midrule
CAPM & 0.012 & 0.002 \\
FF3 & 0.011 & 0.000 \\
CAR & 0.007 & 0.027 \\
FF5 & 0.005 & 0.132 \\
HXZ & 0.001 & 0.777 \\
\bottomrule
\end{tabular}

    \end{tabularx}
\end{table}

\begin{table}[htbp]
    \caption{$\alpha$ test long-short portfolio for in and out of sample with market weighting}
    \begin{tabularx}{\linewidth}{CC}
        \caption*{Sample period }
        \begin{tabular}{lcc}
\toprule
 & $\alpha$ & $Pvalue$ \\
\midrule
CAPM & 0.004 & 0.089 \\
FF3 & 0.002 & 0.346 \\
CAR & 0.000 & 0.911 \\
FF5 & -0.000 & 0.896 \\
HXZ & -0.001 & 0.821 \\
\bottomrule
\end{tabular}

        &
        \caption*{Post-publication period}
        \begin{tabular}{lcc}
\toprule
 & $\alpha$ & $Pvalue$ \\
\midrule
CAPM & 0.005 & 0.115 \\
FF3 & 0.004 & 0.056 \\
CAR & 0.002 & 0.286 \\
FF5 & -0.002 & 0.389 \\
HXZ & -0.003 & 0.198 \\
\bottomrule
\end{tabular}

    \end{tabularx}
\end{table}


\end{enumerate}