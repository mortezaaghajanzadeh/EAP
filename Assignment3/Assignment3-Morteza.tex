\documentclass[hidelinks,12pt]{article}
\usepackage[a4paper,width=150mm,top=25mm,bottom=25mm]{geometry}
\usepackage[utf8]{inputenc}
\usepackage{graphicx}
\usepackage{lipsum}
\usepackage{amssymb}
\usepackage{apacite}
\usepackage{natbib}
\usepackage{hyperref}
\usepackage{float}
\usepackage{ragged2e}
\usepackage[font={footnotesize,bf}]{caption}
\usepackage[nottoc,numbib]{tocbibind}
\usepackage{multirow}
\usepackage{placeins}
\usepackage{booktabs}
\usepackage{hyperref}
\hypersetup{
    colorlinks=true,
    linkcolor=blue,
    filecolor=magenta,      
    urlcolor=cyan,
    pdftitle={Overleaf Example},
    pdfpagemode=FullScreen,
    }


\linespread{1.5}

\begin{document}

\begin{titlepage}
    \begin{center}
        \vspace*{1cm}
        
        
        % \vfill
        
        \large
        \textbf{Empirical Asset Pricing \\ Assignment 4}
            
        
        \normalsize
        Seyyed Morteza Aghajanzadeh \\
        Department of Finance \\
        Stockholm School of Economics
        
        \vfill
            
        
        \vspace{0.8cm}
        \normalsize
        \centering
        2024-01
            
    \end{center}
\end{titlepage}

\section{}
\subsection*{(a)}

The following table shows the summary statistics of the yields, forward rates, and excess returns. I followed the steps that was given in the assignment to calculate the forward rates and excess returns. Based on the results in table \ref{tab:excess_return_describe}, we can see that the mean of the excess returns is positive. The mean of the excess return increases by bond's maturity. The standard deviation of the excess return is also increasing by bond's maturity which means that longer maturity bonds have higher risk.

\begin{table}[htbp]
\centering
\caption{Summary statistics of yields}
\label{tab:yield_describe}
\resizebox{!}{!}{\begin{tabular}{lrrrrr}
\toprule
{} &  y12 &  y24 &  y36 &  y48 &  y60 \\
\midrule
mean & 0.05 & 0.05 & 0.05 & 0.05 & 0.05 \\
std  & 0.03 & 0.03 & 0.03 & 0.03 & 0.03 \\
min  & 0.00 & 0.00 & 0.00 & 0.00 & 0.00 \\
25\%  & 0.02 & 0.02 & 0.03 & 0.03 & 0.03 \\
50\%  & 0.05 & 0.05 & 0.05 & 0.05 & 0.05 \\
75\%  & 0.07 & 0.07 & 0.07 & 0.07 & 0.07 \\
max  & 0.15 & 0.15 & 0.14 & 0.14 & 0.14 \\
\bottomrule
\end{tabular}
}
\end{table}

\begin{table}[htbp]
\centering
\caption{Summary statistics of forward rates}
\label{tab:forward_describe}
\resizebox{!}{!}{\begin{tabular}{lrrrr}
\toprule
{} &  f24 &  f36 &  f48 &  f60 \\
\midrule
mean & 0.05 & 0.05 & 0.06 & 0.06 \\
std  & 0.03 & 0.03 & 0.03 & 0.03 \\
min  & 0.00 & 0.00 & 0.00 & 0.00 \\
25\%  & 0.03 & 0.03 & 0.04 & 0.04 \\
50\%  & 0.05 & 0.05 & 0.05 & 0.06 \\
75\%  & 0.07 & 0.07 & 0.07 & 0.07 \\
max  & 0.15 & 0.14 & 0.14 & 0.14 \\
\bottomrule
\end{tabular}
}
\end{table}

\begin{table}[htbp]
\centering
\caption{Summary statistics of excess returns}
\label{tab:excess_return_describe}
\resizebox{!}{!}{\begin{tabular}{lrrrr}
\toprule
{} &  rx24 &  rx36 &  rx48 &  rx60 \\
\midrule
mean &  0.00 &  0.01 &  0.01 &  0.01 \\
std  &  0.02 &  0.03 &  0.04 &  0.05 \\
min  & -0.05 & -0.09 & -0.12 & -0.15 \\
25\%  & -0.01 & -0.01 & -0.01 & -0.02 \\
50\%  &  0.00 &  0.01 &  0.01 &  0.01 \\
75\%  &  0.01 &  0.02 &  0.03 &  0.04 \\
max  &  0.05 &  0.09 &  0.12 &  0.16 \\
\bottomrule
\end{tabular}
}
\end{table}

\FloatBarrier

\subsection*{(b)}

In order to provide point estimates, standard errors, and tests of the null hypotheses for the regression results, we followed a specific estimation approach. I estimated the standard errors on the bn estimates using a Newey-West standard error estimation method.

The results of the Fama and Bliss regression, presented in Table \ref{tab:fama_bliss}, show that the coefficients are statistically significant different from zero.
Similarly, the results of the Campbell and Shiller regression, presented in Table \ref{tab:campbell_shiller}, indicate that the coefficients are statistically significant different from one. 
On the other hand, the results of the Backus et al. regression, presented in Table \ref{tab:backus_et_al}, show that the coefficients are not statistically significant different from one. 


\begin{table}[htbp]
\centering
\caption{Fama and Bliss regression results}
\label{tab:fama_bliss}
\resizebox{!}{!}{\begin{tabular}{lcccc}
\toprule
{} &  Coefficient &  Standard Deviation &  Test stats \\
Maturity &              &                     &             \\
\midrule
24       &         1.62 &                0.16 &       10.21 \\
36       &         1.91 &                0.18 &       10.60 \\
48       &         2.08 &                0.20 &       10.19 \\
60       &         2.19 &                0.23 &        9.54 \\
\bottomrule
\end{tabular}
}
\end{table}


\begin{table}[htbp]
    \centering
    \caption{Campbell and Shiller regression results}
    \label{tab:campbell_shiller}
    \resizebox{!}{!}{\begin{tabular}{lcccc}
\toprule
{} &  Coefficient &  Standard Deviation &  Test stats \\
Maturity &              &                     &             \\
\midrule
24       &        -0.41 &                0.43 &       -3.30 \\
36       &        -0.70 &                0.46 &       -3.67 \\
48       &        -0.96 &                0.48 &       -4.08 \\
60       &        -1.21 &                0.50 &       -4.46 \\
\bottomrule
\end{tabular}
}
    \end{table}



\begin{table}[htbp]
    \centering
    \caption{Bakus et al. regression results}
    \label{tab:backus_et_al}
    \resizebox{!}{!}{\begin{tabular}{lcccc}
\toprule
{} &  Coefficient &  Standard Deviation &  Test stats \\
Maturity &              &                     &             \\
\midrule
24       &         0.30 &                5.80 &       -0.12 \\
36       &         0.51 &                3.13 &       -0.16 \\
48       &         0.60 &                2.11 &       -0.19 \\
60       &         0.65 &                1.61 &       -0.22 \\
\bottomrule
\end{tabular}
}
    \end{table}


\appendix

\section*{Appendix}

Here you can find the python code that I used to solve the exercise. \href{https://github.com/mortezaaghajanzadeh/BDAP/tree/main/Assignments/Assignment4}{Link to the GitHub repository.}

\end{document}